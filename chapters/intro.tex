%-*-coding: utf-8-*-
\startprefacepage
<<ВКонтакте>> - социальная сеть, большая часть которой разрабатывается на языке PHP, начиная с 2006 года.
В 2013 году суточная посещаемость ВКонтакте достигла почти 50 млн. пользователей\cite{kphp-vk-2013} и социальная сеть столкнулась с проблемой обработки большого числа запросов.
Было принято решение о разработке нового инструмента для ускорения и уменьшения потребления памяти занимаемой работой PHP скрипта.
KPHP был разработан для ускорения, написанного кода, который транслировал PHP код в C++.
По факту не было поддержки ООП и функционального стиля, но этого было достаточно, чтобы перевести сайт на KPHP, ускорив тем самым загрузку страниц почти в два раза.

С того времени было написано много нового кода, внутренняя инфраструктура развивалось и возникала необходимость в поддержке нового синтаксиса и соответствующей функциональности.
В современном мире функциональное программирование занимает важную роль, благодаря которому увеличивается производительность программистов, а код становится более выразительным\cite{fp-matters}.
Хорошо структурированные программы легки в написании, отлаживании, чтении и предоставляют набор модулей, которые с без труда могут быть переиспользованы, что в свою очередь уменьшает стоимость доработок в будущем. Функции высших порядков и анонимные функции улучшают модульность программ.
В KPHP существовала возможность передачи указателей на функции во встроенные методы, но нельзя было передавать указатели в произвольные функции и сохранять их в переменные.
Из-за этого приходилось писать громоздкий код, который было сложно поддерживать.

В связи с тем, что не было возможности в KPHP задания типа принимаемой функции приходилось обобщать возвращаемый тип таких функций как \verb|array_map| до типа-суммы, что ухудшало производительность и увеличивало потребление памяти.
С появлением в языке инстансов классов пропала возможность использования стандартных функций с массивом экземпляров классов, т. к. генерировать все возможные типы-суммы из пользовательских классов и примитивных типов не представляется возможным из-за их количества.
Данные проблемы будут описаны подробнее в первой главе.

Далее будет подробнее рассмотрена поставленная задача и подходы необходимые для успешной поддержки функций высших порядков и анонимных функций в языке KPHP.
Будет представлен созданный синтаксис для типизации внутренних функций высших порядков.
Увидим, что потеря типизации ухудшает статический анализ и увеличивает количество ошибок в коде.
Рассмотрим как можно реализовать данную функциональность и не потерять в производительности и потреблении памяти.
Также будет приведен сравнительный анализ других инструментов позволяющих решать эту задачу, например HHVM \cite{hhvm} и PHP7.0 \cite{php7}, а также возможности для дальнейшего развития.
