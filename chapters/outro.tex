% -*-coding: utf-8-*-

\startconclusionpage
В настоящей работе были разработаны лямбда-функции, а также функции высших порядков в языке KPHP.
Был добавлена концепция шаблонных функций, которая позволяет производить мономорфизацию в случае вызова с экземплярами разных классов.
Также был разработан специальный синтаксис для аннотирования встроенных функций, позволяющих проверять типы и правильно типизировать параметры и возвращаемое значение, лямбда-функции, а также результата встроенных функций.
В рамках данной работы были поддержаны интерфейсы для возможности сохранения разных анонимных функций в одну переменную, сейчас данное решение немного многословно, но в будущем будет не сложно, базируясь на данных разработках, поддержать полноценную возможность сохранения лямбда-функций в поля классов.

В данной работе были рассмотрены: устройство компилятора и рантайма языка KPHP.
Показано, как лаконично встроить текущее решение в существующую инфраструктуру.
Был проведен сравнительный анализ реализации предложенного подхода с имеющимся аналогами такими как PHP7.2 и HHVM4.2.
Данное сравнение показало, что проделанная работа не накладывает дополнительных накладных расходов, а также, что превосходит другие решение в несколько раз по времени работы.
Мы увидели, что благодаря данной разработке KPHP стал находить ошибки в коде, связанные с типизацией, лямбда-функций, передаваемых во встроенные функции с чем не справляется ни одно из решений на данный момент.

Данная работа была успешно внедрена и используется в компании ООО <<В Контакте>>.
Сейчас в кодовой базе насчитывается около 742 анонимных функций, 61 лямбда-функций, захватывающих какие-либо переменные.
Также уже написано 55 функций высших порядков.
Были найдены и исправлены ошибки, связанные с типизацией, лямбда-функций.

Таким образом данная разработка нашла свое применение.
Благодаря достигнутым результатам программисты в компании смогли использовать анонимные функции повсеместно, что доказывает количество их использования с момента поддержки.
Также добавление интерфейсов в язык открывает новые горизонты развития языка KPHP и позволяет писать более понятный и простой код разработчикам сайта.

\printmainbibliography
