%-*-coding: utf-8-*-

\startrelatedwork
\chapter{Постановка задачи}
Язык KPHP уже существует более пяти лет за это время образовалась достаточно большая кодовая база.
Сейчас в проекте более 60000 строк кода и почти 30000 строк в файлах, содержащих тесты к нему, по результатам запуска \verb|sloccount| \cite{sloccount}. Так как было решено, что необходимо добавить поддержку функций высших порядков и анонимных функций, важно было придумать лаконичное решение, вписывающиеся в текущую архитектуру.

В данной работе будет показано, как организовать логику передачи анонимных функций и сохранения их в переменные, вписывающуюся в имеющуюся архитектуру, учитывая, что KPHP -- это типизированный язык. Продумать как будут анонимные функции транслироваться в C++ код и что они будут представлять из себя. Каждый вызов анонимной функции необходимо транслировать в соответствующий код вызывающий нужную функцию, который корректно обрабатывает все входные данные и захватит нужное окружение.

Для передачи лямбда-функций, методов и других функций в качестве аргументов можно использовать мономорфизацию, которая отлично вписывается в текущую архитектуру, но вот для записи в одно и тоже поле класса разных объектов лямбда-функций будет сложнее. Для этого необходимо разработать интерфейсы и соответствующий синтаксис для возможности задания функционального типа в коде и сохранение в него различных объектов, которые можно вызывать без потери производительности, которые в свою очередь будут автоматически наследоваться от соответствующего интерфейса. 

Нужно продумать как поддержать ссылочную семантику для интерфейсов.
Добавить возможность задания классов, которые реализуют тот или иной интерфейс, продумать архитектуру для создания наследования интерфейсов друг от друга.
Провести статический анализ на корректность классов, реализующих какие-либо интерфейсы.
Поддержать возможность проверки на подтип \verb|instanceof| и сделать возможным понижающие приведение.
Также нужно поддерживать состояние для различных переменных, экземпляром какого из подтипа она может являться, чтобы автоматически производить понижающее приведение где это необходимо -- это должно значительно упростить использование интерфейсов разработчиками.

Для уменьшения ошибок при написании анонимных функций с захватом внешних переменных разработать синтаксис, который будет ограничивать переменные на запись, добавляя им семантику константных переменных.
Зачастую разработчики не ожидают поведения, предоставляемого языков PHP, когда они пытаются изменить захваченные переменные, а если им нужно будет поменять их, то надо будет скопировать в другую переменную, чтобы наверняка понимать, что захваченные аргументы не меняют своих значений.

Нужно продумать новый синтаксис для типизации встроенных в язык функций таких как \verb|array_map|, \verb|array_reduce| и другие.
Все они принимают анонимную функцию в качестве аргумента, типы аргументов которой зависят от передаваемых параметров во встроенную функцию.
Разработав систему задания типов для анонимных функций, необходим понять, как внедрить это в существующую инфраструктура языка KPHP, для корректного сопоставления с типами во всех местах использования.

\section{Термины и понятия}
\textbf{Трансляция} -- перевод из одного языка программирования в другой с целью ускорения и уменьшения потребления памяти.

\textbf{Лямбда-функция} -- функция, которая не имеет уникального идентификатора и обычно создается в месте использования.

\textbf{Анонимная функция} -- лямбда-функция.

\textbf{Функция высших порядков} -- такая функция, которая либо принимает другую функцию в качестве аргумента, либо возвращает какую-либо функцию.

\textbf{Захваченная переменная} -- переменная, которая была сохранена каким-либо образом и продолжает существовать внутри лямбда-функции.

\textbf{Ссылочная семантика} -- когда любое присвоение объектов в другие переменные копирует только ссылку на него.

\textbf{Понижающее приведение} -- преобразование переменной, ссылающийся на базовый класс, к одному из производных классов.

\textbf{Встроенная функция} -- функция которая написаны на C++, тело которой скрыто для разработчиков на языке KPHP, однако разработчики имеют возможность вызывать ее.

\textbf{Типизация} -- процесс сопоставления различных переменных с соответствующими типами, а также возвращаемых значений функций.

\textbf{Проверка на подтип} -- процесс в результате которого получается значение из булева множества, истинность которого показывает является ли соответствующая переменная данным подтипом.

\textbf{Инстанс} -- один из экземпляров конкретного класса.

\textbf{Мономорфизация} -- техника, которая заключается в порождении мономорфного экземпляра для каждого случая использования полиморфной функции или типа.

\textbf{Примитивный тип} -- такой тип, встроенный в язык KPHP, который может представлять из себя один из типов: целое число, вещественное число, значение из булева множества, строка, а также массив.

\textbf{Тип-сумма} -- тип, построенный как дизъюнктное объедение исходных типов.

\textbf{Рантайм} -- среда выполнения или такое окружение, которое необходимо для выполнения программы. В KPHP - это отдельная подключаемая библиотека.

\textbf{CFG} -- граф потока управления, множество всех возможных путей исполнения программы, представленное в виде графa.

\textbf{AST} -- абстрактное синтаксическое дерево, представляет из себя дерево, каждая вершина которого обозначает конструкцию, встречающуюся в исходном коде.

\textbf{JIT} -- технология увеличения производительности программных систем, вовлекающая компиляцию некоторых частей программы во время исполнения, а не до момента запуска.

\textbf{Манглирование} -- техника присвоения уникального имени различным объектам, встречающимся в программе.

\section{Актуальность}
\label{sec:actuality}
На текущий момент существует несколько способов решения поставленной задачи.
Один из простых вариантов, который можно рассмотреть -- это отказ от использования KPHP в пользу других языков программирования.
Радикально менять язык программирования будет достаточно большой и трудоемкой задачей, которая вовлечет за собой большие денежные затраты со стороны компании и вряд ли оправдает свою стоимость.
В кодовой базе насчитывается более двух миллионов строк кода, написанных на языке KPHP (совместимым с PHP).
Даже если нанять 100 программистов, которые каждый рабочий день будут переписывать по 100 строчек кода, то у них уйдет на это почти пять месяцев. За это время конечно же появится новый код, а еще нужно будет интегрировать в текущую архитектуру и исправить все допущенные ошибки - на это тоже уйдет достаточно много времени.
Из-за трудоемкости и неоправданной дороговизне данное решение не будет рассматриваться в дальнейшем.

Посмотрим, что еще можем предпринять в данной ситуации. 
Так как код написан на PHP, то кажется разумным решением запускать его с помощью интерпретатора.
Этот подход имеет право на существование, но в следующих главах будет показано, что он также имеет множество недостатков.
Например, KPHP анализирует написанный код, что позволяет находить массу ошибок, допущенных программистами.
Также в KPHP существуют дополнительные возможности, совместимые с языком PHP, для применения дополнительных ограничений на переменные и проверки заданных ограничений до момента запуска.
Основным преимуществом конечно же является скорость работы, которая позволяет уменьшить количество дорогостоящих серверов, что в свою очередь влечет снижение расходов компании на содержание и закупку новых.

Следующим кандидатом может выступать язык Hack и его виртуальная машина HHVM, разработанные в компании facebook.
Данный язык является диалектом PHP, который позволяет совмещать в себе динамическую и статическую типизацию.
У него есть несколько недостатков, из-за которых его использование не подходит компании:

\begin{enumerate}
\item Он не совместим с языком PHP, что вызывает некоторые сложности в его использовании, как минимум нужно будет исправить много кода, а также придется отказаться от возможности использовать стандартные средства для статического анализа, так как все они предназначены для PHP;

\item Среда разработки \verb|PHPStorm| не поддерживает Hack \cite{hack-postponed};

\item Хоть у него и есть JIT \cite{hack-jit}, данный язык с его окружением по-прежнему на многих тестах сильно медленнее KPHP, будет показана в главе \ref{sec:comparision}, что приведет нас к покупке дополнительных серверов и колоссальным затратам на их поддержку;

\item Нужно переделать существующую инфраструктура для развёртывания необходимого программного обеспечения на сервера, что тоже займет не мало времени, в том числе возникнут дополнительные проблемы и спецэффекты, которые на данный момент сложно обнаружить.
\end{enumerate}

Из приведенных аргументов выше можно заключить, что переход на другие языки, даже очень близкие к KPHP трудоемкий на текущем этапе развития компании, то давайте взглянем на несколько примеров, которые могут заменить использование лямбд и функций высших порядков в языке.
Посмотрим, что происходит, если у нас отсутствует возможность сохранять анонимные функции в переменные:
\begin{lstlisting}[caption={Пример кода без анонимных функций},label={without_lambda}]
function debug_for_foo($x) {
  var_dump("in foo: value of x: {$x}");
}

function foo() {
  $x = get_new_x();
  debug_for_foo($x);

  $y = calc_value($x);
  debug_for_foo($y);
}
\end{lstlisting}

В листинге \ref{without_lambda} видно, что небольшой дублирующийся код вместо того, чтобы вынести локально в переменную, содержащую анонимную функцию и тут же его вызывать, вынуждены определять в отдельной функции.
Также на этом примере наглядно продемонстрирована проблема того, что при вынесении всех мелких вспомогательных функций будет необходимо манглировать их имена специальным образом, если их тело будет отличаться в разных функциях, что приводит к загромождению кода и загрязнению глобального пространства имен функций.
С использованием лямбда-функций возможно переписать этот пример следующим образом, что несомненно улучшает читаемость кода и уменьшает затраченное время на его написание:
\begin{lstlisting}
function foo() {
  $debug = function($x) { var_dump("in foo: value of x: {$x}"); };
  $x = get_new_x();
  $debug($x);

  $y = calc_value($x);
  $debug($y);
}
\end{lstlisting}

Теперь взглянем, как можно обойти проблему отсутствия функций высших порядков, с помощью определения нескольких функций, которые будут вызывать другие функции:
\begin{lstlisting}[caption={Пример кода без функций высших порядков},label={without_getting_lambda}]
function debug_result_of_call($result) {
  var_dump("in function foo: value of x: {$x}");
}

function get_x_y_from_net() {
	return [get_x_from_server(), get_y_from_server()];
}

function get_sum($x, $y) {
    return $x + $y;
}

function debug_result_of_call_sum() {
	list($x, $y) = get_x_y_from_net();
	debug_result_of_call(get_sum($x, $y));
}

function get_sub($x, $y) {
    return $x - $y;
}

function debug_result_of_call_sub() {
	list($x, $y) = get_x_y_from_net();
	debug_result_of_call(get_sub($x, $y));
}

function debug_combinations_of_x_y() {
	debug_result_of_call_sum();
	debug_result_of_call_sub();
}
\end{lstlisting}

В листинге \ref{without_getting_lambda} отображены явные недостатки отсутствия функций высших порядков.
Так как необходимо избавляться по максимуму от дублирования кода, программисты будут вынуждены все общие части выносить в отдельные функции.
На этом примере также показана необходимость создания временных имен, для того, чтобы вызывать разные функции и чем больше будет таких примеров, тем больше будет ненужных функций в коде, которые ухудшают читаемость и подвержены ошибкам, допускаемыми при копировании похожих частей кода.
Попробуем реализовать, как надо было бы решать данную проблему используя функции высших порядков:
\begin{lstlisting}
function debug_result(callable $get_result) {
	list(x, y) = [get_x_from_server(), get_y_from_server()];
	var_dump("in function foo: value of x: ".$get_result($x));
}

function debug_combinations_of_x_y() {
	debug_result(function ($x, $y) { return $x + $y; });
	debug_result(function ($x, $y) { return $x - $y; });
}
\end{lstlisting}

Простота и изящность - так можно охарактеризовать, приведенный код выше.
Данный листинг показывает, что писать на языке без лямбда-функций и функций высших порядков довольно таки сложно, а допустить ошибку в коде становится намного легче.
Посмотрим какие еще сложности возникают в текущей момент.
На данном этапе в языке нет синтаксиса, для спецификации типов аргументов лямбда функций и типа возвращаемого значения, поэтому, считается, что все они возвращают некоторый тип-суммы примитивных типов.
Что в свою очередь приводит к тому, что мы не можем их использовать с экземплярами классов, например:
\begin{lstlisting}
class MyClass { public $x = 10; }

$arr = [new MyClass(), new MyClass];
$xs = array_map(function($m) { return $m->x; }, $arr);
\end{lstlisting}

Этот пример показывает отсутствие возможности написать такой код, так как принимаемое значение должно иметь примитивный тип и попытка передать ему экземпляр класса заканчивается неуспехом.
Соответственно все встроенные функции будут запрещены в использовании с классами.
Также использования встроенных функций с лямбда-функциями там, где тип выводится верно, например, с массивом чисел будет ухудшения в производительности, из-за обобщения примитивных типов до типа-суммы.
\finishrelatedwork

\section{Уточненные требования к работе}
В предыдущей секции были рассмотрены примеры и показана необходимость создания поддержки лямбда-функций и функций высших порядков в языке KPHP. Далее будут показаны детализированные этапы, необходимые для совершения данной работы.

Сначала нужно разобраться, что из себя представляют анонимные функции, то есть прочитать соответствующую спецификацию к языку PHP и выделить основные моменты и тонкости синтаксиса.
Необходим написать синтаксический анализ таких функций с дальнейшим построением AST, введя необходимые типы вершин для этого.
Решить во что будет происходить трансляция создания анонимных функций и как будет реализован вызов, также необходимо не забыть поддержать ссылочную семантику при создании лямбда-функций.
Продумать куда сохранять захваченные переменные в теле лямбда-функции, а после необходимо модифицировать ее тело таким образом, чтобы все обращения к захваченным переменным происходили через экземпляр только что созданного класса.
Нужно учитывать, что при определении анонимных функций внутри метода класса, необходимо неявно захватывать \verb|$this|, а также позаботиться о том, что все обращения внутри \verb|$this->field_name| были корректно обработаны.

Следующим этапом будет продумывание семантики для захваченных переменных, как говорилось ранее нужно разработать синтаксис, который бы ограничивал, захваченные переменные на запись.
Соответственно стоит ввести аннотацию \verb|@kphp-const| и соответствующие модификаторы во внутренних структурах, позволяющие проверять на изменения данных переменных.
Также соответственно необходимо спроектировать инфраструктуру для проверки кода, и выдачи ошибок пользователям о произошедшей модификации переменной, которая помечена данной аннотацией.

Для создания функций высших порядков, надо продумать каким образом будет происходить передача лямбда-функции, так как трансляция происходит в типизированный язык, а семантика PHP запрещает перегрузку функций, то необходимо произвести соответствующую мономорфизацию функций, принимающих другие функции.
Соответственно надо разработать специальный синтаксис для поддержки шаблонных функций в языке KPHP, которым смогут пользоваться другие разработчики и по возможности автоматически понимать в каких местах можно аннотировать параметры как шаблонные автоматически.

Также Понадобиться новый синтаксис для типизации передаваемых анонимных функций во встроенные функции.
Надо будет аннотировать разработанным синтаксисом все существующие стандартные функции.
Поддержать их синтаксический анализ и сохранение нужной метаинформации в вершинах AST дерева.
Внедрить вывод типов для случаев передачи лямбд в стандартные функции.
Поддержать корректную трансляцию таких вызовов, а также рассмотреть, как выводить результирующие типы корректно во всех возможных случаях использования.

Для того чтобы появилась возможность сохранять различные лямбда функции, но имеющие схожую сигнатуру в одну и ту же переменную, надо разработать поддержку интерфейсов в язык KPHP.
Для чего необходимо изучить текущую спецификацию и возможные подводные камни.
Разобрать синтаксис, научиться транслировать их.
Также внедрить генерацию виртуальных методов интерфейсов и проверку на нужный подтип.

\chapterconclusion
В данной главе была детально рассмотрена поставленная задача.
Разобрана ее актуальность и необходимость реализации.
Приведены примеры в каких ситуациях без поддержки данной функциональности не получится должного дальнейшего развития языка.
Проведено сравнение с другими решениями из чего можно заключить, что принятое решение о разработке данной задачи, именно в языке KPHP, на данный момент оправдано.

Также были рассмотрены этапы, необходимые для построения архитектуры и дальнейшего анализа, отражающие сложность и многогранность поставленной перед нами задачи.
В следующих главах остановимся подробнее на каждом этапе и посмотрим какие варианты были бы приемлемы для решения и почему были выбраны именно такие варианты.
Также будет показано сравнение разработанного решения с аналогами и показана оправданность и справедливость данных подходов.
